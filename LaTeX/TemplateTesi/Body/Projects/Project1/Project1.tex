\lorem[1]

\subsection*{Details}
\begin{description}
    \item[Authors] \href{http://orcid.org/0000-0002-3397-7737}{\underline{Luca Menestrina}}, \href{https://orcid.org/0000-0001-9518-5412}{Chiara Cabrelle}, \href{http://orcid.org/0000-0002-0039-0518}{Maurizio Recanatini}
    \item[Type] Research Article
    \item[Status] Published
    \item[Title] \href{https://www.nature.com/articles/s41598-021-98812-0}{COVIDrugNet: a network‑based web tool to investigate the drugs currently in clinical trial to contrast COVID‑19}
    \item[Journal] Scientific Reports
    \item[DOI] \href{http://doi.org/10.1038/s41598-021-98812-0}{10.1038/s41598-021-98812-0}
    \item[Data Availability] The full code for the collection, building and analysis of the networks is available in the GitHub repository at \url{https://github.com/LucaMenestrina/COVIDrugNet}. It is entirely written in Python. All data generated or analyzed in this study is publicly available on the GitHub repository. Furthermore, some data is easily downloadable from the web tool itself: all tables in tab-separated values (tsv) format and the networks in various formats (adjacency list, pickle, cytoscape json, graphml, gexf, edges list, multiline adjacency list, tsv, png and jpg).\\
    Supplementary data can also be accessed at the \href{https://www.nature.com/articles/s41598-021-98812-0#Sec19}{original publication}.
\end{description}
\clearpage

\section*{Title}

% \subsection{Abstract}

\subsection{Introduction}
    \lipsum[1-5]

\subsection{Results and Discussion}
    \lipsum[1-10]

\subsection{Limitations}
    \lipsum[1-2]

\subsection{Conclusions}
    \lipsum[1]

\subsection{Methods}
    \lipsum[1-8]